\documentclass[11pt,fancy,authoryear]{elegantbook}

\title{C++ in Detail}
\subtitle{An in-depth guide to C++ on x86 systems with GCC}

\author{Nick Green}
\institute{CoffeeBeforeArch}
\date{December 31, 2020}
\version{0.0}
%\bioinfo{Bio}{Information}

%\extrainfo{Victory won\rq t come to us unless we go to it. }

%\logo{logo-blue.png}
\cover{cover.jpg}

\begin{document}

\maketitle

\frontmatter
\tableofcontents

\mainmatter

\chapter{Introduction}

From databases and web browsers to operating systems and compilers, C++ can be found almost everywhere. This along with a incredibly active community makes it a great time to start learning C++!

\section{Why This Book on C++?}

Out of all the great books on C++, why should you pick this one? Here is a list of reasons!

\begin{itemize}
  \item \textbf{It's free:} This is book (and its source) will always be free to download
  \item \textbf{It can be updated:} Printed books can not be updated. With a digital book, I can make modifications as new standards of C++ get approved, or correct errors in the text.
  \item \textbf{It fills in a gap left by other books:} Many entry level books focus solely on the C++ language, and avoid talking about implementation-specific details. This book breaks this mold by discussing these details which can be increadibly imprortant in practice.
\end{itemize}

\section{Intended Audience}

This book was written as a companion piece to my YouTube series \textbf{C++ Crash Course}. As such, the intended audience is expected to have no prior background in C++ programming.

Having some knowledge of basic programming concepts (e.g., types, functions, etc.) would be useful.

\section{Environment}

\begin{itemize}
  \item \textbf{Operating System: } Ubuntu 20.04
  \item \textbf{Compiler: } GCC 10.2
  \item \textbf{Processor: } Intel x86\-64
\end{itemize}

\section{About the Author}

My name is Nick, and I'm a computer systems architect working on deep learning accelerator performance in the San Francisco Bay Area.

My start in programming came during my undergraduate degree in electrical engineering where I took \textit{Engineering Programming with C}. While I enjoyed this course, I did not take any more dedicated programming classes during this degree (although I did some basic MATLAB and Python programming as required by various coursework).

My start with C++ came during my Ph.D. (incomplete), where I used and developed the GPGPU architecture simulator, GPGPU-Sim. This simulator (largely written in C++), was where I spent a large portion of my time.

After approximately one year of writing C-like C++, I decided to take C++ programming more seriously. There was no single book or resource I used to learn the language. Instead, I would run across a piece of C++ I didn't understand (something like a container or template), search the internet to see out how it worked, then write a minimal example to solidify my understanding. These minimal examples are what eventually became my YouTube series \textbf{C++ Crash Course}.

From then on, C++ and high performance software became one of my passions. I began writing benchmarks in C++, writing blog posts, and doing teaching livestreams.

\chapter{Foundation}

Before we dive into code samples, we need to first discuss some of the foundational parts of C++. This includes the compiler, type system, and functions. It is important to start our journey here, because the compiler, types, and functions play a role in every C++ program.

\section{The Compiler}

\section{Types}

\section{Functions}

\chapter{Basics}

\section{Hello World!}

\section{Working with Variables}

\chapter{Control Flow}

\section{Conditional Statements}

\subsection{If and Else Satements}

\subsection{Switch Statements}

\section{Loops}

\subsection{For Loops}

\subsection{While and Do-While Loops}

\section{Functions}

\subsection{Basics}

\subsection{Overloading}

\subsection{Templates}

\chapter{Memory}

\section{Pointers and References}

\section{Dynamic Memory Allocation}

\chapter{Objects}

\section{Classes and Structs}

\subsection{Constructors}

\subsection{Destructors}

\subsection{Copy Constructors}

\subsection{Operators}

\section{Templates}

\section{Inheritance}

\section{Polymorphism}

\chapter{STL Containers and Algorithms}

\end{document}
