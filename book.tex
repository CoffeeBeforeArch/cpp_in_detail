\documentclass[11pt,fancy,authoryear]{elegantbook}

\title{C++ in Detail}
\subtitle{An in-depth guide to C++ on x86 systems with GCC}

\author{Nick Green}
\institute{CoffeeBeforeArch}
\date{December 31, 2020}
\version{0.0}
%\bioinfo{Bio}{Information}

%\extrainfo{Victory won\rq t come to us unless we go to it. }

%\logo{logo-blue.png}
\cover{cover.jpg}

\begin{document}

\maketitle

\frontmatter
\tableofcontents

\mainmatter

\chapter{Introduction}

From databases and web browsers to operating systems and compilers, C++ can be found almost everywhere. This along with a incredibly active community makes it a great time to start learning C++!

\section{Why This Book on C++?}

Out of all the great books on C++, why should you pick this one? Here is a list of reasons!

\begin{itemize}
  \item \textbf{It's free:} This is book (and its source) will always be free to download
  \item \textbf{It can be updated:} Printed books can not be updated. With a digital book, I can make modifications as new standards of C++ get approved, or correct errors in the text.
  \item \textbf{It fills in a gap left by other books:} Many entry level books focus solely on the C++ language, and avoid talking about implementation-specific details. This book breaks this mold by discussing these details which can be increadibly imprortant in practice.
\end{itemize}

\section{Intended Audience}

This book was written as a companion piece to my YouTube series \textbf{C++ Crash Course}. As such, the intended audience is expected to have no prior background in C++ programming.

Having some knowledge of basic programming concepts (e.g., types, functions, etc.) would be useful.

\section{Environment}

\begin{itemize}
  \item \textbf{Operating System: } Ubuntu 20.04
  \item \textbf{Compiler: } GCC 10.2
  \item \textbf{Processor: } Intel x86\-64
\end{itemize}

\section{About the Author}

My name is Nick, and I'm a computer systems architect working on deep learning accelerator performance in the San Francisco Bay Area.

My start in programming came during my undergraduate degree in electrical engineering where I took \textit{Engineering Programming with C}. While I enjoyed this course, I did not take any more dedicated programming classes during this degree (although I did some basic MATLAB and Python programming as required by various coursework).

My start with C++ came during my Ph.D. (incomplete), where I used and developed the GPGPU architecture simulator, GPGPU-Sim. This simulator (largely written in C++), was where I spent a large portion of my time.

After approximately one year of writing C-like C++, I decided to take C++ programming more seriously. There was no single book or resource I used to learn the language. Instead, I would run across a piece of C++ I didn't understand (something like a container or template), search the internet to see out how it worked, then write a minimal example to solidify my understanding. These minimal examples are what eventually became my YouTube series \textbf{C++ Crash Course}.

From then on, C++ and high performance software became one of my passions. I began writing benchmarks in C++, writing blog posts, and doing teaching livestreams.

\chapter{Foundation}

Before we dive into code samples, we need to first discuss some of the foundational parts of C++. This includes the compiler, type system, and functions. It is important to start our journey here, because the compiler, types, and functions play a role in every C++ program.

\section{The Compiler}

Processors can not directly read/execute high-level C++ code. It first must be translated into the instructions that the processor can understand (the processor's instruction set). This is done with another program known as the compiler.

The purpose of the compiler is to translate code in some source language into a target language. For the majority of examples in this book, we will focus on using the compiler to translate source code written in C++, to machine code that can execute on an x86 processor.

\section{The C++ Type System}

At the lowest level, every part of a program (from the instructions to the data) is composed of bits. How we interpret those bits is determined by types. Types in C++ fall into two categories: fundamental types, and compound types.

\subsection{Fundamental Types}

Fundamental types (sometimes referred to as built-in types) are those types are are built into the compiler, and do not require the inclusion of any other source files to use. These can be broken into the void type, integral types, and floating-point types.

\subsubsection{The Void Type}

Void is a type with an empty set of values. This means that no value can be of the void type. As no value can be of type void, what is it used for? 

In practice, void is used to express a lack of data (e.g., to express no data is returned from or passed to a function).

\subsubsection{The Boolean Type}

The boolean type, \lstinline{bool}, is capable of holding one of two values: \lstinline{true} and lstinline{false}.

\subsubsection{Integral Types}

Integral types are used to represent whole numbers that can be either signed (positive or negative) or unsigned (only positive). The basic form of this type is the \lstinline{int}.

\subsubsection{Floating-Point Types}

Floating-point types are used to represent decimal numbers according to the IEEE-754 floating point standard. The two core types in this category are \lstinline{float} (32-bits) and \lstinline{double} (64-bits).

\subsection{Compound Types}

All types that are not categorized as fundamental are considered to be compound types. According to the C++ standard, compound types can be constructed in the following ways:

\begin{itemize}
  \item Arrays of objects of a given type
  \item Functions which have parameters of given types and return void or references or objects of a given type
  \item Pointers to cv void or objects or functions (including static members of classes) of a given type
  \item References to objects or functions of a given type
  \item Classes
  \item Unions
  \item Enumerations
  \item Pointers to non-static class members
\end{itemize}

Two intuitive ways we can think about compound types are based on where the type comes from, and what they are made of. Compound types are often user-defined (i.e. not part of the compiler). Compound types are also typically built on other types. For example, \lstinline{int} is a fundamental type, but we take the address of the integer (where it exists in memory), we get a pointer to an integer (a compound type, based on the type \lstinline{int}).

\section{Functions}

Functions allow us to associate a name with a sequence of statements. At their core, a function has four parts:

\begin{itemize}
  \item A return type
  \item A name
  \item An argument list
  \item A function body
\end{itemize}

\chapter{Basics}

\section{Hello World!}

\section{Working with Variables}

\chapter{Control Flow}

\section{Conditional Statements}

\subsection{If and Else Satements}

\subsection{Switch Statements}

\section{Loops}

\subsection{For Loops}

\subsection{While and Do-While Loops}

\section{Functions}

\subsection{Basics}

\subsection{Overloading}

\subsection{Templates}

\chapter{Memory}

\section{Pointers and References}

\section{Dynamic Memory Allocation}

\chapter{Objects}

\section{Classes and Structs}

\subsection{Constructors}

\subsection{Destructors}

\subsection{Copy Constructors}

\subsection{Operators}

\section{Templates}

\section{Inheritance}

\section{Polymorphism}

\chapter{STL Containers and Algorithms}

\end{document}
